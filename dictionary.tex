%%------------------------------------------------
%% Begin Koma Article
%%------------------------------------------------
\documentclass[
    DIV=12,
    BCOR=0mm,
    pagenumber=off,
    paper=11in:8.5in,
    pagesize,
]{scrartcl}

\usepackage{anyfontsize}
\usepackage{enumitem}
\usepackage{fouriernc}
\usepackage{booktabs}
%\usepackage{paralist}
\usepackage{siunitx}
\usepackage{tabu}
\usepackage{tipa}
        
\setlength{\itemsep}{1pt}
\setlength{\itemindent}{0pt}
\setlength{\leftmargin}{0pt}
\setlength{\parskip}{0pt}
\setlength{\parsep}{0pt}

%%------------------------------------------------
%% Begin Document
%%------------------------------------------------
\begin{document}

\pagestyle{empty}
%\fontsize{38}{48}\selectfont
\fontsize{32}{36}\selectfont
\raggedright


%% Metrology
%%--------------------
\hfill\vfill
\begin{minipage}{\linewidth}
    \textbf{metrology}
    \textipa{
        /m\textepsilon{}\textprimstress{}t\textturnr{}\textscripta{}.l\textschwa{}.d\textyogh{}i/
    } noun
    \begin{enumerate}
        \item (uncountable) The science of weights and measures or of measurement.
        \item (countable) A system of weights and measures.
    \end{enumerate}
    From Ancient Greek $\mu\acute{\epsilon}\tau\rho{}o\nu$
        (\emph{m\'{e}tron}, ``measure'') + -logy. \\
    \noindent First known use: 1812
\end{minipage}
\vfill
\newpage


%% Measurand
%%--------------------
\hfill\vfill
\begin{minipage}{\linewidth}
    \textbf{measurand}
    \textipa{
        /\textprimstress{}m\textepsilon{}\textyogh{}.\textschwa{}r.\ae{}nd/
    } noun
    \begin{enumerate}
        \item A quantity intended to be measured.
        \item (engineering) An object being measured.
    \end{enumerate}
\end{minipage}
\vfill
\newpage


%% SI
%%--------------------
\hfill\vfill
\begin{minipage}{\linewidth}
    \textbf{SI}
    \textipa{
        /\textepsilon{}s.a\textsci{}/
    } abbreviation
    \begin{enumerate}
        \item International System of Units
    \end{enumerate}
    From French Syst\`{e}me International d'Unit\'{e}s
\end{minipage}
\vfill
\newpage


%% GUM
%%--------------------
\hfill\vfill
\begin{minipage}{\linewidth}
    \textbf{GUM}
    \textipa{
        /g\textturnv{}m/
    } abbreviation
    \begin{enumerate}
        \item Guide to the Expression of Uncertainty in Measurement
    \end{enumerate}
    From French Le Guide pour l'expression de l'incertitude de mesure
\end{minipage}
\vfill
\newpage


%% VIM
%%--------------------
\hfill\vfill
\begin{minipage}{\linewidth}
    \textbf{VIM}
    \textipa{
        /v\textsci{}m/
    } abbreviation
    \begin{enumerate}
        \item International Vocabulary of Metrology
    \end{enumerate}
    From French Vocabulaire international de m\'{e}trologie 
\end{minipage}
\vfill
\newpage


%% Kinematics
%%--------------------
\hfill\vfill
\begin{minipage}{\linewidth}
    \textbf{kinematics}
    \textipa{
        /ki.n\textschwa{}\textprimstress{}m\ae{}.t\textsci{}ks/
    } noun
    \begin{enumerate}
        \item a branch of dynamics that deals with aspects of motion
            apart from considerations of mass and force
    \end{enumerate}
    From Ancient Greek $\kappa\acute{\iota}\nu\eta\mu\alpha$ (k\'{i}n\={e}ma, ``motion'') + -ics. \\
    \noindent First known use: 1840
\end{minipage}
\vfill
\newpage


%% kinetic
%%--------------------
\hfill\vfill
\begin{minipage}{\linewidth}
    \textbf{kinetic}
    \textipa{
        /k\textsci{}n\textprimstress{}\textepsilon{}t\textsci{}k/
        %/k\textschwa{}n"\textepsilon{}t\textsci{}k/
    } adjective
    \begin{enumerate}
        \item Of or relating to motion
        \item Of or relating to kinesis
    \end{enumerate}
    From Ancient Greek $\kappa\iota\nu\eta\tau\iota\kappa\acute{o}\zeta$
        (\emph{kin\={e}tik\'{o}s}, ``one who puts in motion''),
        from $\kappa\iota\nu\acute{\epsilon}\omega$
        (\emph{kin\'{e}\={o}}, ``I move, put in motion'')
\end{minipage}
\vfill
\newpage


%% force
%%--------------------
\hfill\vfill
\begin{minipage}{\linewidth}
    \textbf{force}
    \textipa{
        /fo\textschwa{}\textturnr{}s/
    } noun
    \begin{enumerate}
        %\item Strength or energy of body or mind; active power; vigour; might; capacity of exercising an influence or producing an effect.
        %\item Power exerted against will or consent; compulsory power; violence; coercion.
        %\item (countable) Anything that is able to make a big change in a person or thing.
        \item (countable, physics) A physical quantity that denotes ability to push, pull, twist or accelerate a body which is measured in a unit dimensioned in ($\frac{mass\times{}length}{time^2}$): SI: newton (N); CGS: dyne (dyn)
        %\item Something or anything that has the power to produce an effect upon something else.
        %\item (countable) A group that aims to attack, control, or constrain.
        %\item (uncountable) The ability to attack, control, or constrain.
    \end{enumerate}
    From neuter plural of Latin \emph{fortis} (``strong'').
\end{minipage}
\vfill
\newpage


%% friction
%%--------------------
\hfill\vfill
\begin{minipage}{\linewidth}
    \textbf{friction}
    \textipa{
        /\textprimstress{}f\textturnr{}\textsci{}k\textesh{}\textschwa{}n/
    } noun
    \begin{enumerate}
        \item (physics): A force that resists the relative motion or tendency to such motion of two bodies in contact.
    \end{enumerate}
    From Latin \emph{frictionem}, nominative \emph{frictio} (``a rubbing, rubbing down'')
\end{minipage}
\vfill
\newpage


%% inertia
%%--------------------
\hfill\vfill
\begin{minipage}{\linewidth}
    \textbf{inertia}
    \textipa{
        /\textsci{}n\textprimstress{}\textrhookrevepsilon{}.\textesh{}\textschwa{}/
        /\textsci{}\textprimstress{}n\textrhookrevepsilon{}.\textesh{}\textschwa{}/
    } noun
    \begin{enumerate}
        \item (physics, uncountable or countable) The property of a body that resists any change to its uniform motion; equivalent to its mass.
    \end{enumerate}
    From Latin \emph{iners} (``without skill; inactive''), from \emph{in-} (``not'') + \emph{ars} (``art, skill'').
\end{minipage}
\vfill
\newpage


%% meter
%%--------------------
\hfill\vfill
\begin{minipage}{\linewidth}
    \textbf{meter}
    \textipa{ 
        /\textprimstress{}mit\textschwa{}\textturnr{}/
    } noun
    \begin{enumerate}
        \item (chiefly US, elsewhere metre)
            The base unit of length in the International System of Units (SI),
            conceived of as \num{1/10000000} of the distance from the North Pole to the Equator,
            and now defined as the distance light will travel in a vacuum in \num{1/299792458} second.
            Symbol m.
    \end{enumerate}
    From French m\`{e}tre,
        from Ancient Greek $\mu\acute{\epsilon}\tau\rho{}o\nu$ (m\'{e}tron, ``measure''). \\
\end{minipage}
\vfill
\newpage


%% second
%%--------------------
\hfill\vfill
\begin{minipage}{\linewidth}
    \textbf{second}
    \textipa{ 
        /\textprimstress{}s\textepsilon{}k.(\textschwa{})n?/
        %% ?: glottal stop
    } noun
    \begin{enumerate}
        \item The SI unit of time, defined as the duration of \num{9 192 631 770}
            periods of radiation corresponding to the transition between two
            hyperfine levels of caesium-\num{133} in a ground state at a
            temperature of absolute zero and at rest.
            Symbol: s.
    \end{enumerate}
    From Old French \emph{seconde},
        from Medieval Latin \emph{secunda},
        short for secunda pars minuta (``second diminished part (of the hour)'')
\end{minipage}
\vfill
\newpage


%% kilogram
%%--------------------
\hfill\vfill
\begin{minipage}{\linewidth}
    \textbf{kilogram}
    \textipa{ 
        /\textprimstress{}kilog\textturnr{}\ae{}m/
    } noun
    \begin{enumerate}
        \item In the International System of Units, the base unit of mass;
            conceived of as the mass of one liter of water, and now defined
            as the mass of a specific cylinder of platinum-iridium alloy
            kept at the International Bureau of Weights and Measures in France.
            Symbol: kg.
    \end{enumerate}
    From French \emph{kilogramme}, composed of \emph{kilo}- + \emph{gramme}.
\end{minipage}
\vfill
\newpage


%% dynamics
%%--------------------
\hfill\vfill
\begin{minipage}{\linewidth}
    \textbf{dynamics}
    \textipa{
        /da\textsci{}\textprimstress{}n\ae{}m\textsci{}ks/
    } noun
    \begin{enumerate}
        \item (mechanics) The branch of mechanics that is concerned
            with the effects of forces on the motion of objects.
    \end{enumerate}
    From French \emph{dynamique},
        from Ancient Greek $\delta\upsilon\nu\alpha\mu{}ik\acute{o}\sigma$ (\emph{dunamik\'{o}s}, ``powerful''),
        from $\delta\acute{\upsilon}\nu\alpha\mu{}i\sigma$ (\emph{d\'{u}namis}, ``power''),
        from $\delta\acute{\upsilon}\nu\alpha\mu{}ai$ (\emph{d\'{u}namai}, ``I am able'').
\end{minipage}
\vfill
\newpage


%% ampere
%%--------------------
\hfill\vfill
\begin{minipage}{\linewidth}
    \textbf{ampere}
    \textipa{
        %/da\textsci{}\textprimstress{}n\ae{}m\textsci{}ks/
    } noun
    \begin{enumerate}
        \item A unit of electrical current, the standard base unit in the
            International System of Units. Abbreviation: amp, Symbol: A

        \slshape Definition: The ampere is that constant current which,
            if maintained in two straight parallel conductors of infinite length,
            of negligible circular cross-section, and placed 1 metre apart in vacuum,
            would produce between these conductors a force equal to \SI{1e-7}{\newton\per\meter}
            of length. (The International Bureau of Weights and Measures)
    \end{enumerate}
\end{minipage}
\vfill
\newpage






\end{document}

