%\documentclass[avery5388,grid]{flashcards}
\documentclass[avery5371,grid]{flashcards}

\usepackage{myflashcards}


%% headings, empty, plain
\cardfrontstyle[\Large\slshape]{headings}
\cardbackstyle{empty}

%% left, center, right
%\cardfrontfootstyle[\small\scshape]{right}
%\cardfrontheadstyle[\small\scshape]{left}
\setlength{\cardmargin}{0.25in}
\renewcommand{\cardpaper}{letterpaper}
\renewcommand{\cardpapermode}{portrait}
\renewcommand{\cardrows}{5}
\renewcommand{\cardcolumns}{2}
\setlength{\cardheight}{2.0in}
\setlength{\cardwidth}{3.5in}
\setlength{\topoffset}{0.50in}
\setlength{\oddoffset}{0.75in}
\setlength{\evenoffset}{0.75in}


%All four commands must be defined and all five lengths must be specified.
%If the array of cards is not centered left-to-right on the paper,
%you should set \oddoffset to the left margin of the front and \evenoffset to the right margin of the front

\begin{document}
\cardfrontfoot{Vectors}

%% Basic

\begin{flashcard}[Definition]{What two quantities comprise a vector?}
    Magnitude and direction
\end{flashcard}

\begin{flashcard}[Definition]{What do you do with any vector that is not on either the $x$ or $y$ axis}
    Magnitude and direction
\end{flashcard}

%% Advanced

\begin{flashcard}[Definition]{Dot product of two vectors: $a\cdot b = \ldots$}
    \begin{align*}
        a\cdot b &= a b \cos\theta
        a\cdot b &= a_x b_x + a_y b_y + a_z b_z
    \end{align*}
\end{flashcard}

\begin{flashcard}[Definition]{Cross product of two vectors: $a\times b = \ldots$}
    \begin{align*}
        a \times b &= a b \sin\theta
    \end{align*}
\end{flashcard}

\end{document}

