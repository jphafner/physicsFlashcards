%\documentclass[avery5388,grid]{flashcards}
\documentclass[avery5371,grid]{flashcards}

\usepackage{myflashcard}

%% left, center, right
%\cardfrontfootstyle[\small\scshape]{right}
%\cardfrontheadstyle[\small\scshape]{left}
\setlength{\cardmargin}{0.25in}
\renewcommand{\cardpaper}{letterpaper}
\renewcommand{\cardpapermode}{portrait}
\renewcommand{\cardrows}{5}
\renewcommand{\cardcolumns}{2}
\setlength{\cardheight}{2.0in}
\setlength{\cardwidth}{3.5in}
\setlength{\topoffset}{0.50in}
\setlength{\oddoffset}{0.75in}
\setlength{\evenoffset}{0.75in}


%All four commands must be defined and all five lengths must be specified.
%If the array of cards is not centered left-to-right on the paper,
%you should set \oddoffset to the left margin of the front and \evenoffset to the right margin of the front

\begin{document}
\cardfrontfoot{Dynamics}


%% Section: Copryright
%%------------------------------
\begin{flashcard}[Copyright \& License]{Copyright \copyright \, 2016 Jeffrey Hafner}
\vspace{\fill}
This work is licensed under a Creative Commons Attribution 4.0 International License.
\vspace{\fill}
\flushright
\small\url{https://creativecommons.org/licenses/by/4.0/}
\end{flashcard}


%% Section: Definitions
%%------------------------------
\begin{flashcard}[Definition]{Newton's First Law}
\vspace{\fill}
Every body perseveres in its state of rest, or of uniform motion in a right line, unless it is compelled to change that state by forces impressed thereon.\cite{NaturalPhilosophy}
\vspace{\fill}
\end{flashcard}

\begin{flashcard}[Definition]{Newton's Second Law}
\vspace{\fill}
The alteration of motion is ever proportional to the motive force impressed; and is made in the direction of the right line in which that force is impressed.\cite{NaturalPhilosophy}
\vspace{\fill}
\end{flashcard}


\begin{flashcard}[Definition]{Newton's Third Law}
\vspace{\fill}
To every action there is always opposed an equal reaction: or the mutual actions of two bodies upon each other are always equal, and directed to contrary parts.\cite{NaturalPhilosophy}
\vspace{\fill}
\end{flashcard}


%% Section: Application
%%------------------------------
\begin{flashcard}[Application]{Newton's First Corollary}
\vspace{\fill}
\begin{minipage}{0.60\linewidth}
    A body by two forces conjoined will describe the diagonal of a parallelogram, in the same time that it would describe the sides, by those forces apart.\cite{NaturalPhilosophy}
\end{minipage}
\begin{minipage}{0.35\linewidth}
\begin{center}
\begin{tikzpicture}
    \path (0:1.5) ++ (60:2) coordinate (X);
    \draw[thick,-latex] (0,0) -- (0:1.5) node[pos=0.5,anchor=north] {$A$};
    \draw[thick,-latex] (0,0) -- (60:2) node[pos=0.5,anchor=south east] {$B$};
    \draw[thick,-latex] (0,0) -- (X) node[pos=0.5,anchor=south,rotate=37] {$A+B$};
    \draw[dashed] (0:1.5) -- (X);
    \draw[dashed] (60:2) -- (X);
\end{tikzpicture}
\end{center}
\vspace{\fill}
\end{minipage}
\end{flashcard}


\begin{flashcard}[Application]{Newton's Second Corollary}
\vspace{\fill}
\begin{minipage}{0.60\linewidth}
    And hence is explained the composition of any one direct force AD, out of any two oblique forces AC and CD; and, on the contrary, the resolution of any one direct force AD into two oblique forces AC and CD: which composition and resolution are abundantly confirmed from mechanics.\cite{NaturalPhilosophy}
\end{minipage}
\begin{minipage}{0.38\linewidth}
\vspace{\fill}
\begin{center}
\begin{tikzpicture}
    \coordinate (A) at (60:2);
    \coordinate (C) at (0,0);
    \coordinate (D) at (0:1.5);
    \path (60:2) ++ (0:1.5) coordinate (B);
    %% nodes
    \node[anchor=south] at (A) {$A$};
    \node[anchor=south] at (B) {$B$};
    \node[anchor=north] at (C) {$C$};
    \node[anchor=north] at (D) {$D$};
    %% draw
    \draw (A) -- (B);
    \draw (A) -- (C);
    \draw (A) -- (D);
    \draw (D) -- (B);
    \draw (D) -- (C);
\end{tikzpicture}
\end{center}
\vspace{\fill}
\end{minipage}
\end{flashcard}


\begin{flashcard}[Application]{Newton's Third Corollary}
\vspace{\fill}
The quantity of motion, which is collected by taking the sum of the motions directed towards the same parts, and the difference of those that are directed to contrary parts, suffers no change from the action of bodies among themselves.\cite{NaturalPhilosophy}
\vspace{\fill}
\end{flashcard}


\begin{flashcard}[Application]{Newton's Forth Corollary}
\vspace{\fill}
The common centre of gravity of two or more bodies does not alter its state of motion or rest by the actions of the bodies among themselves; and therefore the common centre of gravity of all bodies acting upon each other (excluding outward actions and impediments) is either at rest, or moves uniformly in a right line.\cite{NaturalPhilosophy}
\vspace{\fill}
\end{flashcard}


\begin{flashcard}[Application]{Newton's Fifth Corollary}
\vspace{\fill}
The motions of bodies included in a given space are the same among themselves, whether that space is at rest, or moves uniformly forwards in a right line without any circular motion.\cite{NaturalPhilosophy}
\vspace{\fill}
\end{flashcard}


\begin{flashcard}[Application]{Newton's Sixth Corollary}
\vspace{\fill}
If bodies, any how moved among themselves, are urged in the direction of parallel lines by equal accelerative forces, they will all continue to move among themselves, after the same, manner as if they had been urged by no such forces.\cite{NaturalPhilosophy}
\vspace{\fill}
\end{flashcard}



\begin{flashcard}[Application]{}
\vspace{\fill}

\vspace{\fill}
\end{flashcard}



%% Section: References
%%------------------------------
\begin{flashcard}[References]{Bibliography}
    \printbibliography[]
\end{flashcard}


\end{document}

