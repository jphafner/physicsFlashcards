%\documentclass[avery5388,grid]{flashcards}
\documentclass[avery5371,grid]{flashcards}

\usepackage{myflashcard}

%% left, center, right
%\cardfrontfootstyle[\small\scshape]{right}
%\cardfrontheadstyle[\small\scshape]{left}
\setlength{\cardmargin}{0.25in}
\renewcommand{\cardpaper}{letterpaper}
\renewcommand{\cardpapermode}{portrait}
\renewcommand{\cardrows}{5}
\renewcommand{\cardcolumns}{2}
\setlength{\cardheight}{2.0in}
\setlength{\cardwidth}{3.5in}
\setlength{\topoffset}{0.50in}
\setlength{\oddoffset}{0.75in}
\setlength{\evenoffset}{0.75in}


%All four commands must be defined and all five lengths must be specified.
%If the array of cards is not centered left-to-right on the paper,
%you should set \oddoffset to the left margin of the front and \evenoffset to the right margin of the front

\begin{document}
\cardfrontfoot{kinematics}


%% Section: Copryright
%%------------------------------
\begin{flashcard}[Copyright \& License]{Copyright \copyright \, 2016 Jeffrey Hafner}
\vspace{\fill}
This work is licensed under a Creative Commons Attribution 4.0 International License.
\vspace{\fill}
\flushright
\small\url{https://creativecommons.org/licenses/by/4.0/}
\end{flashcard}


%% Section: Definitions
%%------------------------------
\begin{flashcard}[Definition]{Position}
\begin{minipage}{0.6\linewidth}
    The location of a point particle along its coordinate axis.
    It is represented as the ordinate or dependent variable in the point particle's space-time coordinate.

    \vspace{\baselineskip}
    \begin{tikzpicture}[x=\textwidth]
        \draw[thick,<->] (0,0) -- (1,0);
        \fill (0.75,0) circle (2pt) node[anchor=north] {$x$};
    \end{tikzpicture}
\end{minipage}
\begin{minipage}{0.35\linewidth}
    \begin{center}
    \begin{tikzpicture}
        \draw[thick,->] (0,0) -- (2,0) node[anchor=north] {$t$};
        \draw[thick,->] (0,0) -- (0,2) node[anchor=east] {$x$};
        \draw (0,0) parabola (2,2);
    \end{tikzpicture}
    \end{center}
\end{minipage}
\end{flashcard}

\begin{flashcard}[Definition]{Time}
\begin{minipage}{0.6\linewidth}
    Time is assumed to be independent of position and to flow uniformly forward.
    It is represented as the abscissa or independent variable in the point particle's space-time coordinate.

    \vspace{\baselineskip}
    \begin{tikzpicture}[x=\textwidth]
        \draw[thick,->] (0,0) -- (1,0);
        \fill (0.75,0) circle (2pt) node[anchor=north] {$t$};
    \end{tikzpicture}
\end{minipage}
\begin{minipage}{0.35\linewidth}
    \begin{center}
    \begin{tikzpicture}
        \draw[thick,->] (0,0) -- (2,0) node[anchor=north] {$t$};
        \draw[thick,->] (0,0) -- (0,2) node[anchor=east] {$x$};
        \draw (0,0) parabola (2,2);
    \end{tikzpicture}
    \end{center}
\end{minipage}
\end{flashcard}

\begin{flashcard}[Definition]{Displacement}
\begin{minipage}{0.6\linewidth}
    The displacement $d$ is the change in position between time $t_f$ and time $t_i$.
        and has dimension of meter (\si{\meter}).
    \begin{displaymath}
        \vec{d} \equiv x_{f} - x_{i}
    \end{displaymath}
    \begin{tikzpicture}[x=\textwidth]
        \draw[thick,->] (0,0) -- (1,0);
        \draw[thick,->] (0.25,0.3) -- (0.75,0.3) node[pos=0.5,anchor=south] {$\vec{d}$};
        \fill (0.25,0) circle (2pt) node[anchor=north] {$x_i$};
        \fill (0.75,0) circle (2pt) node[anchor=north] {$x_f$};
    \end{tikzpicture}
\end{minipage}
\begin{minipage}{0.35\linewidth}
    \begin{center}
    \begin{tikzpicture}
        \draw[thick,->] (0,0) -- (2,0) node[anchor=north] {$t$};
        \draw[thick,->] (0,0) -- (0,2) node[anchor=east] {$x$};
        \draw (0,0) parabola (2,2);
        \draw[dashed] (0.75,0) -- (0.75,0.281) node[pos=0,anchor=north] {$t_i$};
        \draw[dashed] (1.5,0) -- (1.5,1.125) node[pos=0,anchor=north] {$t_f$};
        \draw[dashed] (0,0.281) -- (0.5,0.281) node[pos=0,anchor=east] {$x_i$};
        \draw[dashed] (0,1.125) -- (1.5,1.125) node[pos=0,anchor=east] {$x_f$};
    \end{tikzpicture}
    \end{center}
\end{minipage}
\end{flashcard}

\begin{flashcard}[Definition]{Velocity}
\begin{minipage}{0.6\linewidth}
    Velocity $v$ is the rate change in position,
        and has dimension of meter per second (\si{\meter\per\second}).
    \begin{displaymath}
        \vec{v} \equiv \frac{\Delta x}{\Delta t}
    \end{displaymath}
\end{minipage}
\begin{minipage}{0.35\linewidth}
    \begin{center}
    \begin{tikzpicture}
        \draw[thick,->] (0,0) -- (2,0) node[anchor=north] {$t$};
        \draw[thick,->] (0,0) -- (0,2) node[anchor=east] {$x$};
        \draw (0,0) -- (2,2);
        \draw[dashed] (0.75,0.75) -- (1.5,0.75) node[pos=0.5,anchor=north] {$\Delta t$};
        \draw[dashed] (1.5,0.75) -- (1.5,1.5) node[pos=0.5,anchor=west] {$\Delta x$};
    \end{tikzpicture}
    \end{center}
\end{minipage}
\end{flashcard}

\begin{flashcard}[Definition]{Acceleration}
\begin{minipage}{0.6\linewidth}
    Acceleration $a$ is the rate change in velocity,
        and has dimension of meter per square second (\si{\meter\per\square\second}).
    \begin{displaymath}
        \vec{a} \equiv \frac{\Delta v}{\Delta t}
    \end{displaymath}
\end{minipage}
\begin{minipage}{0.35\linewidth}
    \begin{center}
    \begin{tikzpicture}
        \draw[thick,->] (0,0) -- (2,0) node[anchor=north] {$t$};
        \draw[thick,->] (0,0) -- (0,2) node[anchor=east] {$v$};
        \draw (0,0) -- (2,2);
        \draw[dashed] (0.75,0.75) -- (1.5,0.75) node[pos=0.5,anchor=north] {$\Delta t$};
        \draw[dashed] (1.5,0.75) -- (1.5,1.5) node[pos=0.5,anchor=west] {$\Delta v$};
    \end{tikzpicture}
    \end{center}
\end{minipage}
\end{flashcard}

%\begin{flashcard}[Definition]{Jerk}
%\begin{minipage}{0.6\linewidth}
%    Jerk $j$ is the rate change in acceleration
%        and has dimension of meter per cubic second (\si{\meter\per\cubic\second}).
%    \begin{displaymath}
%        \vec{j} \equiv \frac{\Delta a}{\Delta t}
%    \end{displaymath}
%\end{minipage}
%\begin{minipage}{0.35\linewidth}
%    \begin{center}
%    \begin{tikzpicture}
%        \draw[thick,->] (0,0) -- (2,0) node[anchor=north] {$t$};
%        \draw[thick,->] (0,0) -- (0,2) node[anchor=east] {$x$};
%        \draw (0,0) parabola (2,2);
%        \draw[dashed] (0.75,0) -- (0.75,0.281) node[pos=0,anchor=north] {$t_i$};
%        \draw[dashed] (1.5,0) -- (1.5,1.125) node[pos=0,anchor=north] {$t_f$};
%        \draw[dashed] (0,0.281) -- (0.5,0.281) node[pos=0,anchor=east] {$x_i$};
%        \draw[dashed] (0,1.125) -- (1.5,1.125) node[pos=0,anchor=east] {$x_f$};
%    \end{tikzpicture}
%    \end{center}
%\end{minipage}
%\end{flashcard}
%
%\begin{flashcard}[Definition]{Jounce (also snap)}
%\begin{minipage}{0.6\linewidth}
%    jounce (also snap) $s$ is the rate change in jerk,
%        and has dimension of meter per quartic second (\si{\meter\per\quartic\second}).
%    \begin{displaymath}
%        \vec{s} \equiv \frac{\Delta j}{\Delta t}
%    \end{displaymath}
%\end{minipage}
%\begin{minipage}{0.35\linewidth}
%    \begin{center}
%    \begin{tikzpicture}
%        \draw[thick,->] (0,0) -- (2,0) node[anchor=north] {$t$};
%        \draw[thick,->] (0,0) -- (0,2) node[anchor=east] {$x$};
%        \draw (0,0) parabola (2,2);
%        \draw[dashed] (0.75,0) -- (0.75,0.281) node[pos=0,anchor=north] {$t_i$};
%        \draw[dashed] (1.5,0) -- (1.5,1.125) node[pos=0,anchor=north] {$t_f$};
%        \draw[dashed] (0,0.281) -- (0.5,0.281) node[pos=0,anchor=east] {$x_i$};
%        \draw[dashed] (0,1.125) -- (1.5,1.125) node[pos=0,anchor=east] {$x_f$};
%    \end{tikzpicture}
%    \end{center}
%\end{minipage}
%\end{flashcard}



%% Section: Application
%%------------------------------

%% Application 1
\begin{flashcard}[Application]{Displacement from Velocity}
\vspace{\fill}
\begin{minipage}{0.6\linewidth}
    Displacement, or change in position, from time $t_i$ to time $t_f$,
        can be calculated from the sum under a velocity curve.
    \begin{displaymath}
        d \equiv \Delta x = \text{area of shaded}
    \end{displaymath}
\end{minipage}
\begin{minipage}{0.35\linewidth}
    \begin{center}
    \begin{tikzpicture}
        \draw[thick,->] (0,0) -- (2,0) node[anchor=north] {$t$};
        \draw[thick,->] (0,0) -- (0,2) node[anchor=east] {$v$};
        %\draw (0,0) parabola (2,2);
        \draw (0,0) -- (2,2);
        \draw[pattern=north east lines] (0.75,0) -- (0.75,0.75) -- (1.5,1.5) -- (1.5,0) -- cycle;
        \draw[dashed] (0.75,0.75) -- (0,0.75) node[anchor=east] {$v_i$};
        \draw[dashed] (1.5,1.5) -- (0,1.5) node[anchor=east] {$v_f$};
        \draw[dashed] (0.75,0.75) -- (0.75,0) node[anchor=north] {$t_i$};
        \draw[dashed] (1.5,1.5) -- (1.5,0) node[anchor=north] {$t_f$};
    \end{tikzpicture}
    \end{center}
\end{minipage}
\vspace{\fill}
\end{flashcard}

%% Application 1
\begin{flashcard}[Application]{Change in Velocity from Acceleration}
\vspace{\fill}
\begin{minipage}{0.6\linewidth}
    The change in velocity, from time $t_i$ to time $t_f$,
        can be calculated from the sum under an acceleration curve.
    \begin{displaymath}
        \Delta v = \text{area of shaded}
    \end{displaymath}
\end{minipage}
\begin{minipage}{0.35\linewidth}
    \begin{center}
    \begin{tikzpicture}
        \draw[thick,->] (0,0) -- (2,0) node[anchor=north] {$t$};
        \draw[thick,->] (0,0) -- (0,2) node[anchor=east] {$a$};
        %\draw (0,0) parabola (2,2);
        \draw (0,0) -- (2,2);
        \draw[pattern=north east lines] (0.75,0) -- (0.75,0.75) -- (1.5,1.5) -- (1.5,0) -- cycle;
        \draw[dashed] (0.75,0.75) -- (0,0.75) node[anchor=east] {$a_i$};
        \draw[dashed] (1.5,1.5) -- (0,1.5) node[anchor=east] {$a_f$};
        \draw[dashed] (0.75,0.75) -- (0.75,0) node[anchor=north] {$t_i$};
        \draw[dashed] (1.5,1.5) -- (1.5,0) node[anchor=north] {$t_f$};
    \end{tikzpicture}
    \end{center}
\end{minipage}
\vspace{\fill}
\end{flashcard}

%% Equation 1
\begin{flashcard}[Application]{The four kinematic equations}
\vspace{\fill}
\begin{minipage}{0.4\linewidth}
    \begin{align*}
        d  &= v_i t + \frac{1}{2} a t^2 \\
        d  &= \bar{v} t \\
        v_f^2 &= v_i^2 + 2 a d \\
        v_f   &= v_i + a t \\
    \end{align*}
\end{minipage}
\begin{minipage}{0.5\linewidth}
    The assumption of constant acceleration is necessary
\end{minipage}
\vspace{\fill}
\end{flashcard}


%% Graph application 0
\begin{flashcard}[Application]{
    \vspace{-\baselineskip}
    \begin{center}
    \begin{tikzpicture}
        \begin{scope}[xshift=-2.5cm]
            \draw[thick,->] (0,0) -- (2,0) node[anchor=west] {$t$};
            \draw[thick,->] (0,0) -- (0,1.5) node[anchor=south] {$x$};
            \node[anchor=center] at (1,0.75) {\huge ?};
        \end{scope}
        \begin{scope}[xshift=0.0cm]
            \draw[thick,->] (0,0) -- (2,0) node[anchor=west] {$t$};
            \draw[thick,->] (0,0) -- (0,1.5) node[anchor=south] {$v$};
            \draw (0,0) -- (2,2);
        \end{scope}
        \begin{scope}[xshift=+2.5cm]
            \draw[thick,->] (0,0) -- (2,0) node[anchor=west] {$t$};
            \draw[thick,->] (0,0) -- (0,1.5) node[anchor=south] {$a$};
            \node[anchor=center] at (1,0.75) {\huge ?};
        \end{scope}
    \end{tikzpicture}
    \end{center}
}
\begin{center}
\begin{tikzpicture}
    \begin{scope}[xshift=-2.5cm]
        \draw[thick,->] (0,0) -- (2,0) node[anchor=west] {$t$};
        \draw[thick,->] (0,0) -- (0,1.5) node[anchor=south] {$x$};
        \draw (0,0) plot[domain=0:2] (\x,{0.375*\x*\x});
    \end{scope}
    \begin{scope}[xshift=0.0cm]
        \draw[thick,->] (0,0) -- (2,0) node[anchor=west] {$t$};
        \draw[thick,->] (0,0) -- (0,1.5) node[anchor=south] {$v$};
        \draw (0,0) -- (2,1.5);
    \end{scope}
    \begin{scope}[xshift=+2.5cm]
        \draw[thick,->] (0,0) -- (2,0) node[anchor=west] {$t$};
        \draw[thick,->] (0,0) -- (0,1.5) node[anchor=south] {$a$};
        \draw (0,1) -- (2,1);
    \end{scope}
\end{tikzpicture}
\end{center}
\end{flashcard}

%% Graph application 1
\begin{flashcard}[Application]{
    \vspace{-\baselineskip}
    \begin{center}
    \begin{tikzpicture}
        \begin{scope}[xshift=-2.5cm]
            \draw[thick,->] (0,0) -- (2,0) node[anchor=west] {$t$};
            \draw[thick,->] (0,0) -- (0,1.5) node[anchor=south] {$x$};
            \node[anchor=center] at (1,0.75) {\huge ?};
        \end{scope}
        \begin{scope}[xshift=0.0cm]
            \draw[thick,->] (0,0) -- (2,0) node[anchor=west] {$t$};
            \draw[thick,->] (0,0) -- (0,1.5) node[anchor=south] {$v$};
            \node[anchor=center] at (1,0.75) {\huge ?};
        \end{scope}
        \begin{scope}[xshift=+2.5cm]
            \draw[thick,->] (0,0) -- (2,0) node[anchor=west] {$t$};
            \draw[thick,->] (0,0) -- (0,1.5) node[anchor=south] {$a$};
            \draw (0,1) -- (2,1);
        \end{scope}
    \end{tikzpicture}
    \end{center}
}
\begin{center}
\begin{tikzpicture}
    \begin{scope}[xshift=-2.5cm]
        \draw[thick,->] (0,0) -- (2,0) node[anchor=west] {$t$};
        \draw[thick,->] (0,0) -- (0,1.5) node[anchor=south] {$x$};
        \draw (0,0) plot[domain=0:2] (\x,{0.375*\x*\x});
    \end{scope}
    \begin{scope}[xshift=0.0cm]
        \draw[thick,->] (0,0) -- (2,0) node[anchor=west] {$t$};
        \draw[thick,->] (0,0) -- (0,1.5) node[anchor=south] {$v$};
        \draw (0,0) -- (2,1.5);
    \end{scope}
    \begin{scope}[xshift=+2.5cm]
        \draw[thick,->] (0,0) -- (2,0) node[anchor=west] {$t$};
        \draw[thick,->] (0,0) -- (0,1.5) node[anchor=south] {$a$};
        \draw (0,1) -- (2,1);
    \end{scope}
\end{tikzpicture}
\end{center}
\end{flashcard}

%% Graph application 2
\begin{flashcard}[Application]{
    \vspace{-\baselineskip}
    \begin{center}
    \begin{tikzpicture}
        \begin{scope}[xshift=-2.5cm]
            \draw[thick,->] (0,0) -- (2,0) node[anchor=west] {$t$};
            \draw[thick,->] (0,0) -- (0,1.5) node[anchor=south] {$x$};
            \draw (0,0) plot[domain=0:2] (\x,{0.375*\x*\x});
        \end{scope}
        \begin{scope}[xshift=0.0cm]
            \draw[thick,->] (0,0) -- (2,0) node[anchor=west] {$t$};
            \draw[thick,->] (0,0) -- (0,1.5) node[anchor=south] {$v$};
            \node[anchor=center] at (1,0.75) {\huge ?};
        \end{scope}
        \begin{scope}[xshift=+2.5cm]
            \draw[thick,->] (0,0) -- (2,0) node[anchor=west] {$t$};
            \draw[thick,->] (0,0) -- (0,1.5) node[anchor=south] {$a$};
            \node[anchor=center] at (1,0.75) {\huge ?};
        \end{scope}
    \end{tikzpicture}
    \end{center}
}
\begin{center}
\begin{tikzpicture}
    \begin{scope}[xshift=-2.5cm]
        \draw[thick,->] (0,0) -- (2,0) node[anchor=west] {$t$};
        \draw[thick,->] (0,0) -- (0,1.5) node[anchor=south] {$x$};
        \draw (0,0) plot[domain=0:2] (\x,{0.375*\x*\x});
    \end{scope}
    \begin{scope}[xshift=0.0cm]
        \draw[thick,->] (0,0) -- (2,0) node[anchor=west] {$t$};
        \draw[thick,->] (0,0) -- (0,1.5) node[anchor=south] {$v$};
        \draw (0,0) -- (2,1.5);
    \end{scope}
    \begin{scope}[xshift=+2.5cm]
        \draw[thick,->] (0,0) -- (2,0) node[anchor=west] {$t$};
        \draw[thick,->] (0,0) -- (0,1.5) node[anchor=south] {$a$};
        \draw (0,1) -- (2,1);
    \end{scope}
\end{tikzpicture}
\end{center}
\end{flashcard}


%% Graph application 3
\begin{flashcard}[Application]{
    \vspace{-\baselineskip}
    \begin{center}
    \begin{tikzpicture}
        \begin{scope}[xshift=-2.5cm]
            \draw[thick,->] (0,0) -- (2,0) node[anchor=west] {$t$};
            \draw[thick,->] (0,0) -- (0,1.5) node[anchor=south] {$x$};
            \node[anchor=center] at (1,0.75) {\huge ?};
        \end{scope}
        \begin{scope}[xshift=0.0cm]
            \draw[thick,->] (0,0) -- (2,0) node[anchor=west] {$t$};
            \draw[thick,->] (0,0) -- (0,1.5) node[anchor=south] {$v$};
            \draw (0,0) plot[domain=0:2] (\x,{0.375*\x*\x});
        \end{scope}
        \begin{scope}[xshift=+2.5cm]
            \draw[thick,->] (0,0) -- (2,0) node[anchor=west] {$t$};
            \draw[thick,->] (0,0) -- (0,1.5) node[anchor=south] {$a$};
            \node[anchor=center] at (1,0.75) {\huge ?};
        \end{scope}
    \end{tikzpicture}
    \end{center}
}
\begin{center}
\begin{tikzpicture}
    \begin{scope}[xshift=-2.5cm]
        \draw[thick,->] (0,0) -- (2,0) node[anchor=west] {$t$};
        \draw[thick,->] (0,0) -- (0,1.5) node[anchor=south] {$x$};
        \draw (0,0) plot[domain=0:2] (\x,{0.1875*\x*\x*\x});
    \end{scope}
    \begin{scope}[xshift=0.0cm]
        \draw[thick,->] (0,0) -- (2,0) node[anchor=west] {$t$};
        \draw[thick,->] (0,0) -- (0,1.5) node[anchor=south] {$v$};
        \draw (0,0) plot[domain=0:2] (\x,{0.375*\x*\x});
    \end{scope}
    \begin{scope}[xshift=+2.5cm]
        \draw[thick,->] (0,0) -- (2,0) node[anchor=west] {$t$};
        \draw[thick,->] (0,0) -- (0,1.5) node[anchor=south] {$a$};
        \draw (0,0) -- (2,1.5);
    \end{scope}
\end{tikzpicture}
\end{center}
\end{flashcard}





\begin{comment}
\begin{flashcard}[Definition]{Average velocity}
\bigskip
\bigskip
\begin{displaymath}
    v_{avg} \equiv \frac{\Delta x}{\Delta t} 
\end{displaymath}
\textbf{or} \\
\medskip
\begin{displaymath}
    \textrm{Average velocity} = \frac{\textrm{change in position}}{\textrm{change in time}} = \frac{\textrm{displacement}}{\textrm{time interval}}
\end{displaymath}
\end{flashcard}

\begin{flashcard}[Definition]{Average speed}
\bigskip
\bigskip
\begin{displaymath}
    \textrm{Average speed} = \frac{\textrm{total distance}}{\textrm{total time}}
\end{displaymath}
\end{flashcard}

\begin{flashcard}[Definition]{Instantaneous velocity}
\bigskip
\bigskip
\begin{displaymath}
    v_{x} \equiv \lim_{\Delta t \rightarrow 0} \frac{\Delta x}{\Delta t} = \frac{dx}{dt}
\end{displaymath}
\end{flashcard}

\begin{flashcard}[Definition]{Average acceleration}
\bigskip
\bigskip
\begin{displaymath}
    \bar{a}_{x} \equiv \frac{\Delta v_{x}}{\Delta t} = \frac{v_{xf} - v_{xi}}{t_{f} - t_{i}}
\end{displaymath}
\end{flashcard}

\begin{flashcard}[Definition]{Instantaneous acceleration}
\bigskip
\bigskip
\begin{displaymath}
    a_{x} \equiv \lim_{\Delta t \rightarrow 0} \frac{\Delta v_{x}}{\Delta t} = \frac{dv_{x}}{dt}
\end{displaymath}
\end{flashcard}

\begin{flashcard}[Definition]{Velocity as a function of time}
\bigskip
\begin{displaymath}
    v_{xf} = v_{xi} + a_{x}t
\end{displaymath}
\begin{center}
    assuming constant acceleration
\end{center}
\end{flashcard}

\begin{flashcard}[Definition]{Position as a function of velocity and time}
\bigskip
\begin{displaymath}
    x_{f} = x_{i} + \frac{1}{2}\left( v_{xi} + v_{xf}\right)t
\end{displaymath}
\begin{center}
    assuming constant acceleration
\end{center}
\end{flashcard}

\begin{flashcard}[Definition]{Position as a function of time}
\bigskip
\begin{displaymath}
    x_{f} = x_{i} + v_{xi}t + \frac{1}{2}a_{x}t
\end{displaymath}
\begin{center}
    assuming constant acceleration
\end{center}
\end{flashcard}

\begin{flashcard}[Definition]{Velocity as a function of position}
\bigskip
\begin{displaymath}
v_{xf}^{2} = v_{xi}^{2} + 2a_{x}\left( x_{f} - x_{i}\right) 
\end{displaymath}
\begin{center}
(constant acceleration)
\end{center}
\end{flashcard}
\end{comment}



%% Section: References
%%------------------------------
\begin{flashcard}[]{Bibliography}
    \printbibliography[title=]
\end{flashcard}


\end{document}

