%\documentclass[avery5388,grid]{flashcards}
\documentclass[avery5371,grid]{flashcards}

\usepackage{myflashcard}

%% headings, empty, plain
\cardfrontstyle[\Large\slshape]{headings}
\cardbackstyle{empty}

\begin{document}
\cardfrontfoot{Measurement}


%%------------------------------
%% Type: Definition
%%------------------------------


%% Section: Significant Digits
%%------------------------------
\begin{flashcard}[Defintion]{Buckingham $\Pi$ Theorem}
\vspace{\fill}

\vspace{\fill}
\end{flashcard}

\begin{flashcard}[Defintion]{significant figures}
\hfill\vfill
\begin{quote}
    attribute of a phenomenon, body or substance that may be distinguished qualitatively and determined quantitatively
\end{quote}
\vfill
\end{flashcard}

\begin{flashcard}[Defintion]{significant figures: addition}
\hfill\vfill
\begin{quote}
    attribute of a phenomenon, body or substance that may be distinguished qualitatively and determined quantitatively
\end{quote}
\vfill
\end{flashcard}

\begin{flashcard}[Defintion]{significant figures: multiplication}
\hfill\vfill
\begin{quote}
    attribute of a phenomenon, body or substance that may be distinguished qualitatively and determined quantitatively
\end{quote}
\vfill
\end{flashcard}


%% Section: SI Units
%%------------------------------
\begin{flashcard}[Defintion]{Defined Unit}
\vfill
\footnotesize\flushright \url{http://www.physics.nist.gov/cuu/Units/units.html}
\end{flashcard}

\begin{flashcard}[Defintion]{Derived Unit}
\vfill
\footnotesize\flushright \url{http://www.physics.nist.gov/cuu/Units/units.html}
\end{flashcard}

\begin{flashcard}[Defintion]{Operational Definition}
\end{flashcard}

\begin{flashcard}[Defintion]{Seven Base SI units}
\vfill
\begin{center}
\begin{tabular}{llc}
    Base Quantity & Name & symbol \\
    \midrule
    length  & meter     & m \\
    mass    & kilogram  & kg \\
    time    & second    & s \\
    electric current            & ampere    & A \\
    thermodynamic temperature   & kelvin    & K \\
    amount of substance         & mole      & mol \\
    luminous intensity          & candela   & cd \\
    \bottomrule
\end{tabular}
\end{center}
\vfill
\end{flashcard}

\begin{flashcard}[Defintion]{the meter (m)}
\vspace{\fill}
The meter is the length of the path travelled by light in vacuum during a time interval of 1/299\,792\,458 of a second.
\vspace{\fill}
\end{flashcard}

\begin{flashcard}[Defintion]{the kilogram (kg)}
\vspace{\fill}
The kilogram is the unit of mass; it is equal to the mass of the international prototype of the kilogram.
\vspace{\fill}
\end{flashcard}

\begin{flashcard}[Defintion]{the second (s)}
\vspace{\fill}
The second is the duration of \num{9 192 631 770} periods of the radiation corresponding to the transition between the two hyperfine levels of the ground state of the cesium-133 atom.
\vspace{\fill}
\end{flashcard}

\begin{flashcard}[Defintion]{the ampere (A)}
\vspace{\fill}
The ampere is that constant current which, if maintained in two straight parallel conductors of infinite length, of negligible circular cross section, and placed 1 meter apart in vacuum, would produce between these conductors a force equal to \num{2e-7} newton per meter of length.
\vspace{\fill}
\end{flashcard}

\begin{flashcard}[Defintion]{the kelvin (K)}
\vspace{\fill}
The kelvin, unit of thermodynamic temperature, is the fraction 1/273.16 of the thermodynamic temperature of the triple point of water.
\vspace{\fill}
\end{flashcard}

\begin{flashcard}[Defintion]{the mole (mol)}
\vspace{\fill}
\begin{enumerate}
    \item  The mole is the amount of substance of a system which contains as many elementary entities as there are atoms in 0.012 kilogram of carbon 12.
    \item  When the mole is used, the elementary entities must be specified and may be atoms, molecules, ions, electrons, other particles, or specified groups of such particles.
\end{enumerate}
\vspace{\fill}
\end{flashcard}

\begin{flashcard}[Defintion]{the candela (cd)}
\vspace{\fill}
The candela is the luminous intensity, in a given direction, of a source that emits monochromatic radiation of frequency \num{540e12} hertz and that has a radiant intensity in that direction of (1/683) watt per steradian.
\vspace{\fill}
\end{flashcard}


%%------------------------------
%% Type: Application
%%------------------------------


%% Section: Significant Digits
%%------------------------------
\begin{flashcard}[Application]{$\SI{1.35}{\meter} + \SI{1.5319}{\meter} =$}
\vspace{\fill}
The candela is the luminous intensity, in a given direction, of a source that emits monochromatic radiation of frequency \num{540e12} hertz and that has a radiant intensity in that direction of (1/683) watt per steradian.
\vspace{\fill}
\end{flashcard}



\end{document}

