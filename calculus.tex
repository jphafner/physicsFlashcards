%\documentclass[avery5388,grid]{flashcards}
\documentclass[avery5371,grid]{flashcards}

\usepackage{myflashcards}


%% headings, empty, plain
\cardfrontstyle[\Large\slshape]{headings}
\cardbackstyle{empty}

%% left, center, right
%\cardfrontfootstyle[\small\scshape]{right}
%\cardfrontheadstyle[\small\scshape]{left}
\setlength{\cardmargin}{0.25in}
\renewcommand{\cardpaper}{letterpaper}
\renewcommand{\cardpapermode}{portrait}
\renewcommand{\cardrows}{5}
\renewcommand{\cardcolumns}{2}
\setlength{\cardheight}{2.0in}
\setlength{\cardwidth}{3.5in}
\setlength{\topoffset}{0.50in}
\setlength{\oddoffset}{0.75in}
\setlength{\evenoffset}{0.75in}


%All four commands must be defined and all five lengths must be specified.
%If the array of cards is not centered left-to-right on the paper,
%you should set \oddoffset to the left margin of the front and \evenoffset to the right margin of the front

\begin{document}
\cardfrontfoot{Calculus}

\begin{flashcard}[Definition]{Graph of $y=\cos x$}
    \begin{tikzpicture}
        \begin{axis}[ ]
            \addplot[ ] { };
        \end{axis}
    \end{tikzpicture}
\end{flashcard}

\begin{flashcard}[Definition]{Graph of $y=\sin x$}
    \begin{tikzpicture}
        \begin{axis}[ ]
            \addplot[ ] { };
        \end{axis}
    \end{tikzpicture}
\end{flashcard}

\begin{flashcard}[Definition]{Graph of $y=\tan x$}
    \begin{tikzpicture}
        \begin{axis}[ ]
            \addplot[ ] { };
        \end{axis}
    \end{tikzpicture}
\end{flashcard}

\begin{flashcard}[Definition]{Graph of $y=\dfrac{1}{x}$}
    \begin{tikzpicture}
        \begin{axis}[ ]
            \addplot[ ] { };
        \end{axis}
    \end{tikzpicture}
\end{flashcard}

\begin{flashcard}[Definition]{Graph of $y=\sqrt{x}$}
    \begin{tikzpicture}
        \begin{axis}[ ]
            \addplot[ ] { };
        \end{axis}
    \end{tikzpicture}
\end{flashcard}

\begin{flashcard}[Definition]{Graph of $y=\sqrt{1-x^2}$}
    \begin{tikzpicture}
        \begin{axis}[ ]
            \addplot[ ] { };
        \end{axis}
    \end{tikzpicture}
\end{flashcard}

\begin{flashcard}[Definition]{Graph of $y=|x|$}
    \begin{tikzpicture}
        \begin{axis}[ ]
            \addplot[ ] { };
        \end{axis}
    \end{tikzpicture}
\end{flashcard}

\begin{flashcard}[Definition]{Graph of $y=e^{x}$}
    \begin{tikzpicture}
        \begin{axis}[ ]
            \addplot[ ] { };
        \end{axis}
    \end{tikzpicture}
\end{flashcard}

\begin{flashcard}[Definition]{Graph of $y=\ln{x}$}
    \begin{tikzpicture}
        \begin{axis}[ ]
            \addplot[ ] { };
        \end{axis}
    \end{tikzpicture}
\end{flashcard}


%% Defintion
\begin{flashcard}[Definition]{An even function \ldots}
    \ldots with respect to the $y$-axis,
        like $y=x^2$, $y=\cos x$, or $y=|x|$.
    \begin{equation*}
        f(-x) = f(x)
    \end{equation*}
\end{flashcard}

\begin{flashcard}[Definition]{An odd function \ldots}
    \ldots symmetric with respect to the origin,
        like $y=x^3$, $y=\sin x$, or $y=\tan x$.
    \begin{equation*}
        f(-x) = -f(x)
    \end{equation*}
\end{flashcard}

%% Surface area
\begin{flashcard}[Definition]{Two formulas for the area of a triangle}
    \begin{align*}
        A &= \frac{1}{2}bh \\
        A &= \frac{1}{2} a b \sin C
    \end{align*}
\end{flashcard}

\begin{flashcard}[Definition]{Formula for area of a circle}
    \begin{equation*}
        A = \pi r^2
    \end{equation*}
\end{flashcard}

\begin{flashcard}[Definition]{Formula for the surface area of a sphere}
    \begin{equation*}
        A = 4 \pi r^2
    \end{equation*}
\end{flashcard}

%% Length
\begin{flashcard}[Definition]{Formula for the circumference of a circle}
    \begin{equation*}
        C = 2 \pi r
    \end{equation*}
\end{flashcard}

%% Volume
\begin{flashcard}[Definition]{Formula for the volume of a cyclinder}
    \begin{equation*}
        V = \pi r^2 h
    \end{equation*}
\end{flashcard}

\begin{flashcard}[Definition]{Formula for the volume of a cone}
    \begin{equation*}
        V = \frac{1}{3} \pi r^2 h
    \end{equation*}
\end{flashcard}

\begin{flashcard}[Definition]{Formula for the volume of a sphere}
    \begin{equation*}
        V = \frac{4}{3} \pi r^3
    \end{equation*}
\end{flashcard}

%% Polynomials
\begin{flashcard}[Definition]{Point-slope form of a linear equation}
    \begin{equation*}
        y - y_1 = m \left( x-x_1 \right)
    \end{equation*}
\end{flashcard}

%% Limits
\begin{flashcard}[Definition]{Point-slope form of a linear equation}
    \begin{equation*}
        y - y_1 = m \left( x-x_1 \right)
    \end{equation*}
\end{flashcard}

%% Limits
\begin{flashcard}[Definition]{A tangent line is \ldots}
    \ldots the line through a point on a curve with slope equal
        to the slope of the curve at that point.
    \begin{tikzpicture}
        %% NOTE: insert example
    \end{tikzpicture}
\end{flashcard}

\begin{flashcard}[Definition]{A secant line is \ldots}
    \ldots the line connecting two points on a curve.
    \begin{tikzpicture}
        %% NOTE: insert example
    \end{tikzpicture}
\end{flashcard}

\begin{flashcard}[Definition]{A normal line is \ldots}
    \ldots the line perpendicular to the tangent line at the point of tangency.
    \begin{tikzpicture}
        %% NOTE: insert example
    \end{tikzpicture}
\end{flashcard}

\begin{flashcard}[Definition]{$f(x)$ is continuous at $x=c$ when \ldots}
    \begin{enumerate}
        \item $f(c)$ exists
        \item $\lim_{x\goto c} f(x)$ exists; and
        \item $\lim_{x\goto c} f(x) = f(c)$ 
    \end{enumerate}
\end{flashcard}

\begin{flashcard}[Definition]{limit definition of the derivative of $f(x)$: $f^{\prime}(x)=\ldots$}
    \begin{equation}
        \lim_{\Delta x\goto 0} \frac{f(x+\Delta x) - f(x)}{\Delta x}
    \end{equation}
\end{flashcard}

\begin{flashcard}[Definition]{alternative definition of derivative of $f(x)$ at $x=c$: $f^{\prime}(c)=\ldots$}
    \begin{equation}
        \lim_{\Delta x\goto c} \frac{f(x) - f(c)}{x-c}
    \end{equation}
\end{flashcard}

\begin{flashcard}[Definition]{What does $f^{\prime}(x)$tell you about a function?}
    \begin{enumerate}
        \item slope of curve at a point
        \item slope of tangent line
        \item instantaneous rate of change
    \end{enumerate}
\end{flashcard}

\begin{flashcard}[Definition]{Average rate of change \ldots}
    \begin{equation}
        \frac{\Delta y}{\Delta x} = \frac{ f(b) - f(a) }{b-a}
    \end{equation}
\end{flashcard}

%% Derivatives
\begin{flashcard}[Definition]{Power rule for derivatives}
    \begin{equation}
        n x^{n-1}
    \end{equation}
\end{flashcard}

\begin{flashcard}[Definition]{Product rule for derivatives:
    $\dfrac{d}{dx}\left( f(x) g(x) \right) = \ldots$}
    \begin{equation}
        f^{\prime}(x) g(x) + g^{\prime}(x) f(x)
    \end{equation}
\end{flashcard}

\begin{flashcard}[Definition]{Quotient rule for derivatives:
    $\dfrac{d}{dx}\left( \frac{f(x)}{g(x)} \right) = \ldots$}
    \begin{equation}
        \frac{ g(x)f^{\prime}(x) - f^{\prime}(x) g(x) }{ \left(g(x)\right)^2 }
    \end{equation}
\end{flashcard}

\begin{flashcard}[Definition]{Chain rule for derivatives:
    $\dfrac{d}{dx}\left( f\left( g\left(x\right) \right) \right) = \ldots$}
    \begin{equation}
        f^{\prime}\left( g(x) \right)\cdot g^{\prime}(x)
    \end{equation}
\end{flashcard}

\begin{flashcard}[Definition]{$\dfrac{d}{dx}\sin x = \ldots$}
    \begin{equation}
        \cos x
    \end{equation}
\end{flashcard}

\begin{flashcard}[Definition]{$\dfrac{d}{dx}\cos x = \ldots$}
    \begin{equation}
        -\sin x
    \end{equation}
\end{flashcard}

\begin{flashcard}[Definition]{$\dfrac{d}{dx}\ln x = \ldots$}
    \begin{equation}
        \frac{1}{x}
    \end{equation}
\end{flashcard}

\begin{flashcard}[Definition]{$\dfrac{d}{dx}\log_a x = \ldots$}
    \begin{equation}
        \frac{1}{x \ln a}
    \end{equation}
\end{flashcard}

\begin{flashcard}[Definition]{$\dfrac{d}{dx}\log_a x = \ldots$}
    \begin{equation}
        \frac{1}{x \ln a}
    \end{equation}
\end{flashcard}

\end{document}

