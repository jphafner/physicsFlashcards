%\documentclass[avery5388,grid]{flashcards}
\documentclass[avery5371,grid]{flashcards}

\usepackage{myflashcard}

%% headings, empty, plain
\cardfrontstyle[\Large\slshape]{headings}
\cardbackstyle{empty}


%% left, center, right
%\cardfrontfootstyle[\small\scshape]{right}
%\cardfrontheadstyle[\small\scshape]{left}
\setlength{\cardmargin}{0.25in}
\renewcommand{\cardpaper}{letterpaper}
\renewcommand{\cardpapermode}{portrait}
\renewcommand{\cardrows}{5}
\renewcommand{\cardcolumns}{2}
\setlength{\cardheight}{2.0in}
\setlength{\cardwidth}{3.5in}
\setlength{\topoffset}{0.50in}
\setlength{\oddoffset}{0.75in}
\setlength{\evenoffset}{0.75in}


%All four commands must be defined and all five lengths must be specified.
%If the array of cards is not centered left-to-right on the paper,
%you should set \oddoffset to the left margin of the front and \evenoffset to the right margin of the front

%% Useful links
%% http://www.nist.gov/itl/sed/gsg/uncertainty.cfm


\begin{document}
\cardfrontfoot{Metrology}

%% Metrology: BIPM: Guide to the expression of uncertainty in measurement
%% Annex B: General metrological terms

\begin{flashcard}[Definition]{(measureable) quantity}
\vspace{\fill}
    attribute of a phenomenon, body or substance that may be distinguished qualitatively and determined quantitatively
\vspace{\fill}
\end{flashcard}

\begin{flashcard}[Definition]{value \\ (of a quantity)}
\vspace{\fill}
    magnitude of a particular quantity generally expressed as a unit of measurement multiplied by a number
\vspace{\fill}
\end{flashcard}

\begin{flashcard}[Definition]{true value \\ (of a quantity)}
\vspace{\fill}
    value consistent with the definition of a given particular quantity
\vspace{\fill}
\end{flashcard}

\begin{flashcard}[Definition]{conventional true value \\ (of a quantity)}
\vspace{\fill}
    value attributed to a particular quantity and accepted, sometimes by convention, as having an uncertainty appropriate for a given purpose
\vspace{\fill}
\end{flashcard}

\begin{flashcard}[Definition]{measurement}
\vspace{\fill}
    set of operations having the object of determining a value of a quantity
\vspace{\fill}
\end{flashcard}

\begin{flashcard}[Definition]{method of measurement}
\vspace{\fill}
    logical sequence of operations, described generically, used in the performance of measurements
\vspace{\fill}
\end{flashcard}

\begin{flashcard}[Definition]{measurement procedure}
\vspace{\fill}
    set of operations, described specifically, used in the performance of particular measurements according to a given method
\vspace{\fill}
\end{flashcard}

\begin{flashcard}[Definition]{measurand}
\vspace{\fill}
    particular quantity subject to measurement
\vspace{\fill}
\end{flashcard}

\begin{flashcard}[Definition]{result of a measurement}
\vspace{\fill}
    value attributed to a measurand, obtained by measurement
\vspace{\fill}
\end{flashcard}

\begin{flashcard}[Definition]{uncorrected result}
\vspace{\fill}
    result of a measurement before correction for systematic error
\vspace{\fill}
\end{flashcard}

\begin{flashcard}[Definition]{accuracy of measurement}
\vspace{\fill}
    closeness of the agreement between the result of a measurement and a true value of the measurand
\vspace{\fill}
\end{flashcard}

\begin{flashcard}[Definition]{repeatability \\ (of results of measurements)}
\vspace{\fill}
    closeness of the agreement between the results of successive measurements of the same measurand carried out under the same conditions of measurement
\vspace{\fill}
\end{flashcard}

\begin{flashcard}[Definition]{reproducibility \\ (of results of measurements)}
\vspace{\fill}
    closeness of the agreement between the results of measurements of the same measurand carried out under changed conditions of measurement
\vspace{\fill}
\end{flashcard}

\begin{flashcard}[Definition]{experimental standard deviation}
\vspace{\fill}
    for a series of n measurements of the same measurand, the quantity $s(q_k)$ characterizing the dispersion of the results and given by the formula:
    \begin{equation*}
        s(q_k) = \sqrt{ \frac{1}{n-1} \sum_{j=1}^{n} \left( q_{ij} - \bar{q} \right)^2 }
    \end{equation*}
    $q_k$ being the result of the $k$th measurement and $q$ being the arithmetic mean of the $n$ results considered
\vspace{\fill}
\end{flashcard}

\begin{flashcard}[Definition]{uncertainty \\ (of measurement)}
\vspace{\fill}
    parameter, associated with the result of a measurement, that characterizes the dispersion of the values that could reasonably be attributed to the measurand
\vspace{\fill}
\end{flashcard}

\begin{flashcard}[Definition]{error \\ (of measurement)}
\vspace{\fill}
    result of a measurement minus a true value of the measurand
\vspace{\fill}
\end{flashcard}

\begin{flashcard}[Definition]{relative error}
\vspace{\fill}
    error of measurement divided by a true value of the measurand
\vspace{\fill}
\end{flashcard}

\begin{flashcard}[Definition]{random error}
\vspace{\fill}
    result of a measurement minus the mean that would result from an infinite number of measurements of the same measurand carried out under repeatability conditions
\vspace{\fill}
\end{flashcard}

\begin{flashcard}[Definition]{systematic error}
\vspace{\fill}
    mean that would result from an infinite number of measurements of the same measurand carried out under repeatability conditions minus a true value of the measurand
\vspace{\fill}
\end{flashcard}

\begin{flashcard}[Definition]{correction}
\vspace{\fill}
    value added algebraically to the uncorrected result of a measurement to compensate for systematic error
\vspace{\fill}
\end{flashcard}

\begin{flashcard}[Definition]{correction factor}
\vspace{\fill}
    numerical factor by which the uncorrected result of a measurement is multiplied to compensate for systematic error
\vspace{\fill}
\end{flashcard}

%% Metrology: BIPM: Guide to the expression of uncertainty in measurement
%% Annex C: Basic statistical terms and concepts

\begin{flashcard}[Definition]{probability}
\vspace{\fill}
    a real number in the scale 0 to 1 attached to a random event correction factor
\vspace{\fill}
\end{flashcard}

\begin{flashcard}[Definition]{random variable, variate}
\vspace{\fill}
    a variable that may take any of the values of a specified set of values and with which is associated a probability distribution
\vspace{\fill}
\end{flashcard}

\begin{flashcard}[Definition]{probability distribution \\ (of a random variable)}
\vspace{\fill}
    a function giving the probability that a random variable takes any given value or belongs to a given set of values
\vspace{\fill}
\end{flashcard}

\begin{flashcard}[Definition]{distribution function}
\vspace{\fill}
    a function giving, for every value $x$,
        the probability that the random variable $X$ be less than or equal to $x$:
    \begin{equation*}
        F ( x ) = Pr ( X \leq x )
    \end{equation*}
\vspace{\fill}
\end{flashcard}

\begin{flashcard}[Definition]{probability density function \\ (for a continuous random variable)}
\vspace{\fill}
    the derivative (when it exists) of the distribution function:
    \begin{equation*}
        f(x) = \mathrm{d}F(x)\,\mathrm{d}x
    \end{equation*}
\vspace{\fill}
\end{flashcard}

\begin{flashcard}[Definition]{probability mass function}
\vspace{\fill}
    a function giving, for each value $x_i$ of a discrete random variable $X$,
        the probability $p_i$ that the random variable equals $x_i$:
    \begin{equation*}
        p_i = Pr( X = x_i )
    \end{equation*}
\vspace{\fill}
\end{flashcard}

%\begin{flashcard}[Definition]{expectation of a discrete random variable \\
% (of a random variable or of a probability distribution), \\ expected value, mean}
\begin{flashcard}[Definition]{expectation (expected value, mean) of a discrete random variable (probability distrubution)}
\vspace{\fill}
    For a discrete random variable $X$ taking the values $x_i$ with the probabilities $p_i$, the expectation, if it exists, is:
    \begin{equation*}
        \mu = E(X) = \sum p_i x_i
    \end{equation*}
    the sum being extended over all the values $x_i$ which can be taken by $X$.
\vspace{\fill}
\end{flashcard}

%\begin{flashcard}[Definition]{expectation of a continuous random variable \\
% (of a random variable or of a probability distribution), \\ expected value, mean}
\begin{flashcard}[Definition]{expectation (expected value, mean) of a continuous random variable (probability distrubution)}
\vspace{\fill}
    For a continuous random variable $X$ having the probability density function $f(x)$, the expectation, if it exists, is:
    \begin{equation*}
        \mu = E(X) = x f(x)\,\mathrm{d}x
    \end{equation*}
    the integral being extended over the interval(s) of variation of $X$.
\vspace{\fill}
\end{flashcard}

\begin{flashcard}[Definition]{variance \\ (of a random variable or of a probability distribution)}
\vspace{\fill}
    the expectation of the square of the centred random variable
    \begin{equation*}
        \sigma^2 = V(X) = E\left\{ \left[ X-E(x) \right]^2 \right\}
    \end{equation*}
\vspace{\fill}
\end{flashcard}

\begin{flashcard}[Definition]{standard deviation \\ (of a random variable or of a probability distribution)}
\vspace{\fill}
    the positive square root of the variance:
    \begin{equation*}
        \sigma = \sqrt{ V(X) }
    \end{equation*}
\vspace{\fill}
\end{flashcard}

\begin{flashcard}[Definition]{central moment of order $q$}
\vspace{\fill}
    in a univariate distribution, the expectation of the $q$ th power of the centred random variable $(X−\mu)$:
    \begin{equation*}
        E \left[ \left( X - \mu \right)^q \right]
    \end{equation*}
\vspace{\fill}
\end{flashcard}

\begin{flashcard}[Definition]{normal distribution, Laplace-Gauss distribution}
\vspace{\fill}
    the probability distribution of a continuous random variable $X$,
        the probability density function of which is:
    \begin{equation*}
        f(x) = \frac{1}{\sigma\sqrt{2\pi}} e^{\frac{-1}{2}\left(\frac{x-\mu}{\sigma}\right)^2 }
    \end{equation*}
\vspace{\fill}
\end{flashcard}

\begin{flashcard}[Definition]{characteristic}
\vspace{\fill}
    a property which helps to identify or differentiate between items of a given population
\vspace{\fill}
\end{flashcard}

\begin{flashcard}[Definition]{population}
\vspace{\fill}
    the totality of items under consideration
\vspace{\fill}
\end{flashcard}

\begin{flashcard}[Definition]{arithmetic mean, average}
\vspace{\fill}
    the sum of values divided by the number of values
\vspace{\fill}
\end{flashcard}

\begin{flashcard}[Definition]{variance}
\vspace{\fill}
    a measure of dispersion, which is the sum of the squared deviations of observations from their average divided by one less than the number of observations
\vspace{\fill}
\end{flashcard}

\begin{flashcard}[Definition]{standard deviation}
\vspace{\fill}
    the positive square root of the variancevariance
\vspace{\fill}
\end{flashcard}

\begin{flashcard}[Definition]{central moment of order $q$}
\vspace{\fill}
    in a distribution of a single characteristic,
        the arithmetic mean of the $q$th power of the difference between the observed values and their average $x$:
    \begin{equation*}
        \frac{1}{n} \sum_{i} \left( x_i - \bar{x} \right)^2
    \end{equation*}
    where $n$ is the number of observations
\vspace{\fill}
\end{flashcard}

\begin{flashcard}[Definition]{statistic}
\vspace{\fill}
    a function of the sample random variables
\vspace{\fill}
\end{flashcard}

\begin{flashcard}[Definition]{estimation}
\vspace{\fill}
    the operation of assigning, from the observations in a sample,
        numerical values to the parameters of a distribution chosen as the statistical model of the population from which this sample is taken
\vspace{\fill}
\end{flashcard}

\begin{flashcard}[Definition]{two-sided confidence interval}
\vspace{\fill}
    when $T_1$ and $T_2$ are two functions of the observed values such that,
        $\theta$ being a population parameter to be estimated,
        the probability $Pr( T_1 \leq \theta \leq T_2 )$ is at least equal to $(1-\alpha )$
        [where $(1-\alpha)$ is a fixed number, positive and less than 1],
        the interval between $T_1$ and $T_2$ is a two-sided $(1-\alpha)$ confidence interval for $\theta$
\vspace{\fill}
\end{flashcard}

\begin{flashcard}[Definition]{one-sided confidence interval}
\vspace{\fill}
    when $T$ is a function of the observed values such that,
        $\theta$ being a population parameter to be estimated,
        the probability $\mathrm{Pr}( T\geq\theta )$ [or the probability $\mathrm{Pr}( T\geq\theta )$] is at least equal to $(1-\alpha)$
        [where $(1-\alpha)$ is a fixed number, positive and less than 1],
        the interval from the smallest possible value of $\theta$ up to $T$
        (or the interval from $T$ up to the largest possible value of $\theta$)
        is a one-sided $(1-\alpha)$ confidence interval for $\theta$
\vspace{\fill}
\end{flashcard}

\begin{flashcard}[Definition]{confidence coefficient, \\ confidence level}
\vspace{\fill}
    the value $(1-\alpha)$ of the probability associated with a confidence interval or a statistical coverage interval
\vspace{\fill}
\end{flashcard}

\begin{flashcard}[Definition]{statistical coverage interval}
\vspace{\fill}
    an interval for which it can be stated with a given level of confidence that it contains at least a specified proportion of the population
\vspace{\fill}
\end{flashcard}

\begin{flashcard}[Definition]{degrees of freedom}
\vspace{\fill}
    in general, the number of terms in a sum minus the number of constraints on the terms of the sum
\vspace{\fill}
\end{flashcard}

\end{document}

