%\documentclass[avery5388,grid]{flashcards}
\documentclass[avery5371,grid]{flashcards}

\usepackage{myflashcard}


%% headings, empty, plain
\cardfrontstyle[\Large\slshape]{headings}
\cardbackstyle{empty}


%% left, center, right
%\cardfrontfootstyle[\small\scshape]{right}
%\cardfrontheadstyle[\small\scshape]{left}
\setlength{\cardmargin}{0.25in}
\renewcommand{\cardpaper}{letterpaper}
\renewcommand{\cardpapermode}{portrait}
\renewcommand{\cardrows}{5}
\renewcommand{\cardcolumns}{2}
\setlength{\cardheight}{2.0in}
\setlength{\cardwidth}{3.5in}
\setlength{\topoffset}{0.50in}
\setlength{\oddoffset}{0.75in}
\setlength{\evenoffset}{0.75in}


%All four commands must be defined and all five lengths must be specified.
%If the array of cards is not centered left-to-right on the paper,
%you should set \oddoffset to the left margin of the front and \evenoffset to the right margin of the front

%% Useful links
%% http://www.nist.gov/itl/sed/gsg/uncertainty.cfm


\begin{document}
\cardfrontfoot{Syst\`{e}me International d'Unit\'{e}s}


%% Section: Copryright
%%------------------------------
\begin{flashcard}[Copyright \& License]{Copyright \copyright \, 2016 Jeffrey Hafner}
\vspace{\fill}
This work is licensed under a Creative Commons Attribution 4.0 International License.
\vspace{\fill}
\flushright
\small\url{https://creativecommons.org/licenses/by/4.0/}
\end{flashcard}


%% SI: International System of Units
%%------------------------------------------------------------
\begin{flashcard}[Definitions]{Base Quantities and Dimensions}
\vspace{\fill}\small
\begin{center}
\begin{tabular}{llc}
    \toprule
                  & \multicolumn{2}{c}{Symbol For} \\
    Base Quantity & quantity & dimension \\
    \midrule
    length  & $l$, $x$, $r$              & L \\
    mass    & $m$                        & M \\
    time    & $t$                        & T \\
    electric current            & $I$    & I \\
    thermodynamic temperature   & $T$    & $\Theta$ \\
    amount of substance         & $n$    & N \\
    luminous intensity          & $I_v$  & J \\
    \bottomrule
\end{tabular}
\end{center}
\vspace{\fill}
\end{flashcard}

\begin{flashcard}[Definitions]{Base SI units}
\vspace{\fill}\small
\begin{center}
\begin{tabular}{llc}
    \toprule
                  & \multicolumn{2}{c}{SI Base Unit} \\
    Base Quantity & Name & symbol \\
    \midrule
    length  & meter     & m \\
    mass    & kilogram  & kg \\
    time    & second    & s \\
    electric current            & ampere    & A \\
    thermodynamic temperature   & kelvin    & K \\
    amount of substance         & mole      & mol \\
    luminous intensity          & candela   & cd \\
    \bottomrule
\end{tabular}
\end{center}
\vspace{\fill}
\end{flashcard}

\begin{flashcard}[Definitions]{Base (defined) Unit}
\vspace{\fill}
The SI base quantities form a set of mutually independent dimensions as required by dimensional analysis commonly employed in science and technology.
\vspace{\fill}
\end{flashcard}

\begin{flashcard}[Definitions]{Derived Unit}
\vspace{\fill}
Other quantities, called derived units are defined in terms of the seven base units.
Where the dimension of unit $Q$, a derived unit, can be written in the form given below
\begin{equation*}
    \mathrm{dim} Q = \mathrm{L}^{\alpha} \mathrm{M}^{\beta} \mathrm{T}^{\gamma} \mathrm{I}^{\delta} \Theta^{\epsilon} \mathrm{N}^{\zeta} \mathrm{J}^{\eta} \, .
\end{equation*}
Where $\alpha$, $\beta$, $\gamma$, $\delta$, $\epsilon$, $\zeta$ and $\eta$ are small integers.
\vspace{\fill}
\end{flashcard}

\begin{flashcard}[Definitions]{meter (m)}
\vspace{\fill}
The meter is the length of the path travelled by light in vacuum during a time interval of 1/\num{299 792 458} of a second.\cite{sp330}
\vspace{\fill}
\end{flashcard}

\begin{flashcard}[Definitions]{kilogram (kg)}
\vspace{\fill}
The kilogram is the unit of mass; it is equal to the mass of the international prototype of the kilogram.\cite{sp330}
\vspace{\fill}
\end{flashcard}

\begin{flashcard}[Definitions]{second (s)}
\vspace{\fill}
The second is the duration of \num{9 192 631 770} periods of the radiation corresponding to the transition between the two hyperfine levels of the ground state of the cesium-133 atom.\cite{sp330}
\vspace{\fill}
\end{flashcard}

\begin{flashcard}[Definitions]{ampere (A)}
\vspace{\fill}
The ampere is that constant current which, if maintained in two straight parallel conductors of infinite length, of negligible circular cross section, and placed 1 meter apart in vacuum, would produce between these conductors a force equal to \num{2e-7} newton per meter of length.\cite{sp330}
\vspace{\fill}
\end{flashcard}

\begin{flashcard}[Definitions]{kelvin (K)}
\vspace{\fill}
The kelvin, unit of thermodynamic temperature, is the fraction 1/273.16 of the thermodynamic temperature of the triple point of water.\cite{sp330}
\vspace{\fill}
\end{flashcard}

\begin{flashcard}[Definitions]{mole (mol)}
\vspace{\fill}
\begin{enumerate} %[itemsep=1pt]
    \item  The mole is the amount of substance of a system which contains as many elementary entities as there are atoms in 0.012 kilogram of carbon 12.
    \item  When the mole is used, the elementary entities must be specified and may be atoms, molecules, ions, electrons, other particles, or specified groups of such particles.\cite{sp330}
\end{enumerate}
\vspace{\fill}
\end{flashcard}

\begin{flashcard}[Definitions]{candela (cd)}
\vspace{\fill}
The candela is the luminous intensity, in a given direction, of a source that emits monochromatic radiation of frequency \num{540e12} hertz and that has a radiant intensity in that direction of (1/683) watt per steradian.\cite{sp330}
\vspace{\fill}
\end{flashcard}


%% Style Guide
%%------------------------------------------------------------
\begin{flashcard}[Definitions]{style}
\vspace{\fill}
Style conventions \cite{sp811}
\vspace{\fill}
\end{flashcard}


%% Section: References
%%------------------------------
\begin{flashcard}[]{Bibliography}
    \printbibliography
\end{flashcard}


\end{document}

